\documentclass[fontsize=13pt,a4paper]{article}

% Use UTF-8 encoding in this document. If the LaTeX compiler complains
% use utf8x instead.
\usepackage[utf8]{inputenc}

% American Mathematical Society packages for various
% mathematical symbols and environments taken for granted.
\usepackage{amsmath}
\usepackage{amsfonts}
\usepackage{amssymb}

% Compatible with KOMA script whereas fancyhdr is not
\usepackage{scrlayer-scrpage}
\pagestyle{scrheadings}

\clearscrheadfoot
\lehead[]{Exercise 3 -- Point clouds}
\lohead[]{Exercise 3 -- Point clouds}
\rehead[]{Due to January 17th, a Thursday}
\rohead[]{Due to January 17th, a Thursday}

\begin{document}

\section{Point-clouds}

\subsection{Loading space separated ascii files}

\begin{itemize}
    \item[(i)]
        Load the file \emph{sandhausen\_sample.xyz},
        kindly provided by Katharina Anders, into Octave
        and visualize the point cloud. Color each point in your
        visualization by its position on the z-axis.

    \item[(ii)]
        Write the result of your visualization to file named
        \emph{A1\_result.png}.
\end{itemize}

\subsection{Obtaining a Digital Terrain Model from a Digitial Surface Model}

\begin{itemize}
    \item[(i)] The transformation presented here, obtaining a
        digitial terrain model (DTM), is closely related to the
        non-maximum suppression. For each point, determine whether
        in its neighbourhood it is the lowest point, i.e. has the lowest
        value on the z-axis.

        Let $p, q \in P$ denote two points in a point-cloud $P \in \mathbb{R}^3$.
        Let $\vec{z} = [0, 0, -1]$ be a vector pointing in the direction
        of the ground. Let $\vec{g} = [1, 1, 0]$ be a vector denoting the
        ground plane. Let $\langle \cdot, \cdot \rangle$ denote the
        the dot product, i.e. projection of a vector.
        Let $\lambda$ be a threshold denoting the
        amount of detail you want to extract.
        Then, a point $p$ is in the DTM point-cloud $\hat{P}$ only if the
        following holds.

        \begin{equation}
\hat{P} = \{p \; | \; \forall p, q \in P:
    \|\langle p, \vec{g} \rangle -
        \langle q, \vec{g} \rangle\| < \lambda \textbf{ and }
                    \langle p, \vec{z} \rangle > \langle q, \vec{z} \rangle \}
        \end{equation}

    \item[(ii)] Implement the procedure outlined above. Make sure
        to vectorize your implementation where possible.

    \item[(iii)] Visualize the point cloud, again coloring points by
        their position on the z-axis, after being processed to extract
        the DTM. Experiment with at least 3 values for $\lambda$.
        Save your plots to files named
        \emph{A2\_result1.png, A2\_result2.png and A2\_result3.png}.
\end{itemize}

\newpage

\subsection{Send me Your Results}

\begin{itemize}
    \item [(i)] Compress your results into a zip file
    with the following file structure.

    \begin{verbatim}
        I3DCV_<matriculation number>_Exercise_3.zip
            |
            |- I3DCV_<matriculation number>_Exercise_3
                |
                |- sandhausen_sample.xyz
                |- A1_exercises.m
                |- A1_result.png
                |- A2_exercises.m
                |- A2_result1.png
                |- A2_result2.png
                |- A2_result3.png
    \end{verbatim}

    \item [(ii)] Send me an email containing "[3DCV] Exercise 3"
    in the subject line with your zip file attached. My email address
    is \emph{bartosz.bogacz@iwr.uni-heidelberg.de}.
\end{itemize}

\end{document}
